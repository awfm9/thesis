\documentclass[11pt,a4paper,draft]{article}

\title{Risk Management for Blockchain Consensus}
\author{Max Wolter}
\date{June 2018}

\begin{document}

\maketitle

\begin{abstract}

\end{abstract}

\section{Introduction}

The concept of Bitcoin as a "peer-to-peer electronic cash system" was introduced by an anonymous author under the pseudonym of Satoshi Nakamoto in October 2008. Since then, the potential of the underlying blockchain technology to serve as a trustless mechanism to find consensus on the state of an underlying system has spread to many applications.\\

The main value proposition of a blockchain platform is its ability to ensure the integrity of the consensus in a verifiable and transparent manner. While the software and business ecosystem developed around a blockchain project are susceptible to the same risks as typical software and business projects, the core platform has its own, unique set of risks.\\

This paper explores the various aspects around blockchain consensus systems that are required to guarantee the continued security of the system state.\\

\section{Consensus}

\subsection{Weaknesses}

When identifying risks, the usual approach dictates that we identify assets which are vulnerable to use them as a basis for the analysis. For a blockchain network, this proposition is a difficult one; the network is composed by thousands of nodes which communicate in a peer-to-peer fashion. While each node can be compromised in its own right, there is no single tangible asset or actor that impacts the security of the network state as a whole.\\

However, we could define the consensus itself as an abstract concept which needs to be protected. This leads us to the first family of weaknesses, where we look for flaws in the theoretical model used for the blockchain consensus. It requires exploration of the possible exploits in the fields of mathematics, more specifically game theory, and information science, in particular distributed systems.\\

Beyond these obvious flaws, we can consider the subject from a more comprehensive perspective. Indeed, the blockchain consensus is composed of three aspects: the algorithm implementing the theoretical model for the consensus, the parties which participate in the forming of the consensus and the governance mechanisms which dictate the rules of the consensus.\\

This means that a blockchain consensus platform could be attacked even if the theoretical model was perfectly secure. On the technical level, threats exist that are specific to blockchain platforms and exploits could be used to disrupt the access to the state in its full integrity.\\

Finally, the project governing a blockchain project can be disrupted in a number of ways, including a number of social engineering and manipulation vectors. This could lead to a situation where the rules of the consensus no longer follow the vision of the majority, leading from an exodus of the users from the platform and putting into question the consensus validity as a whole.\\

The paper will use these three different attack categories to differentiate the possible attacks on the consensus: theoretical attacks, technical attacks and social attacks.\\

\subsection{Impact}

When analysing traditional information technology assets, we usually evaluate the severity of a risk in terms of the impact it has on confidentiality, integrity and availability. In a blockchain network, these three aspects can not be directly translated; some aspects are theoretically guaranteed by blockchain technology, while others are simply not applicable.\\

We therefore propose an alternative system of classification for the impact of consensus failures or disruptions on blockchain networks: accessibility, validity and reliability.\\

Accessibility describes the ability of a participant of the blockchain consensus network to access the correct system state. Validity describes whether the consensus state that is recognized by the network participants is valid. Reliability describes the confidence that the currently known consensus state will remain stable over time.\\

The impacts of a risk in the context of our risk management framework will thus be described in these three categories.\\

\section{Theoretical}

51\% Attack

Selfish Mining

Long Range Attack

(Censorship, Economic Equilibrium, Double Spend)


\section{Technical}

Eclipse Attack

Partition Attack

Delay Attack

Illegal Material

\section{Social}

\subsection{Community Sentiment}

\subsection{Software Code}

\subsection{Project Roadmap}

\section{Interdependencies}

\section{Case Study}

\subsection{Analysis}

\subsection{Evaluation}

\subsection{Mitigation}

\section{Conclusion}

\cite{item}

\begin{thebibliography}{9}

\bibitem{item}
author,
"name",
link,
date.

\end{thebibliography}

\end{document}
