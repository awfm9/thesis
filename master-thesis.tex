\documentclass[11pt,a4paper,draft]{article}

\title{Risk Management for Blockchain Consensus}
\author{Max Wolter}
\date{June 2018}

\begin{document}

\maketitle

\begin{abstract}

\end{abstract}

\section{Introduction}

The concept of Bitcoin as a "peer-to-peer electronic cash system" was introduced by an anonymous author under the pseudonym of Satoshi Nakamoto in October 2008. Since then, the potential of the underlying blockchain technology to serve as a trustless mechanism to find consensus on the state of an underlying system has spread to many applications.\\

The main value proposition of a blockchain platform is its ability to ensure the integrity of the consensus in a verifiable and transparent manner. While the software and business ecosystem developed around a blockchain project are susceptible to the same risks as typical software and business projects, the core platform has its own, unique set of risks.\\

This paper explores the various aspects of consensus that are required to guarantee the continued security of the system state on a blockchain network.\\

\section{Consensus}

When identifying risks, the usual approach dictates that we identify the assets which are vulnerable and use them as a basis for our analysis. For a blockchain network, this proposition is a difficult one; the network is composed by thousands of nodes which communicate in a peer-to-peer fashion. While each node can be compromised in its own right, there is no single tangible asset or actor that impacts the security of network state as a wholie.\\

Rather, we could say that the consensus as an abstract concept is the asset that needs to be protected. As a first observation, we are therefore trying to secure a theoretical model against possible flaws. This necessitates the mathematical exploration of weaknesses in the model and the discovery of ways to safeguard against them.\\

However, if we consider the subject from a more comprehensive perspective, the consensus is not only made up of a game theoretical model. In order for the formed consensus to be meaningful, we also need the rules encoded into the software to stay close to the original intention and we need to make sure that all participants who so desire can participate in the consensus.\\

It is thus possible to disrupt the consensus of a blockchain platform on a more abstract level by getting participants to drop out of the consensus-forming network or by interfering with the project governance in a way that the rules encoded in the software break with the desires of the consensus majority. This leaves blockchain networks open to a number of out-of-band or social attacks.\\

In the following, we categorize possible attacks on the consensus into one of three categories: attacks targeting the game theory of the consensus algorithm, attacks which target the governance of the consensus algorithm rules and attacks which target the participation in the consensus algorithm execution.\\

\section{Game Theory}

\subsection{Censorship}

51\% Attack

\subsection{Economic Equilibrium}

Selfish Mining

\subsection{Double Spend Attack}

Long Range Attack

\section{Participation}

\subsection{Single Node}

Eclipse Attack

\subsection{Group of Nodes}

Partition Attack

Delay Attack

\subsection{Demographics}

Illegal Material Attack

\section{Governance}

\subsection{Community Sentiment}

\subsection{Software Code}

\subsection{Project Roadmap}

\section{Theoretical}

\subsection{Threats}

\subsection{Impacts}

\subsection{Controls}

\section{Organizational}

\subsection{Threats}

\subsection{Impacts}

\subsection{Controls}

\section{Social}

\subsection{Threats}

\subsection{Impacts}

\subsection{Controls}

\section{Interdependencies}

\section{Case Study}

\subsection{Analysis}

\subsection{Evaluation}

\subsection{Mitigation}

\section{Conclusion}

\cite{item}

\begin{thebibliography}{9}

\bibitem{item}
author,
"name",
link,
date.

\end{thebibliography}

\end{document}
