\documentclass[11pt,a4paper,draft]{article}

\usepackage[utf8]{inputenc}
\usepackage[english]{babel}
\usepackage[nottoc,numbib]{tocbibind}

\setcounter{secnumdepth}{2}
\setcounter{tocdepth}{2}

\title{Risk Management for Blockchain Consensus}
\author{Max Wolter}
\date{June 2018}

\begin{document}

\maketitle
\begin{abstract}
\end{abstract}
\newpage

\tableofcontents
\newpage

\section{Introduction}

The concept of Bitcoin as a "peer-to-peer electronic cash system" was introduced by an anonymous author under the pseudonym of Satoshi Nakamoto in October 2008 \cite{bitcoin}. Since then, the potential of the underlying blockchain technology to serve as a trustless mechanism to find consensus on the state of an underlying system has spread to many applications.\\

The main value proposition of a blockchain platform is its ability to ensure the integrity of the consensus in a verifiable and transparent manner. While the software and business ecosystem developed around a blockchain project are susceptible to the same risks as typical software and business projects, the core platform has its own, unique set of risks.\\

This paper explores the various aspects around blockchain consensus systems that are required to guarantee the continued security of the system state.\\

\section{Consensus}

\subsection{Asset}

When identifying risks, the usual approach dictates that we identify assets which are vulnerable and use them as a basis for the analysis. For a blockchain network, this proposition is a difficult one; the network is composed by thousands of nodes which communicate in a peer-to-peer fashion. While each node can be compromised in its own right, there is no single tangible asset or actor that impacts the security of the network state as a whole.\\

However, we could define the consensus itself as an abstract concept which needs to be protected. This leads us to the first family of weaknesses, where we look for flaws in the theoretical model used for the blockchain consensus. It requires exploration of the possible exploits in the fields of mathematics, more specifically game theory, and information science, in particular distributed systems.\\

As the possible vulnerabilities and attacks change with the type of consensus algorithm, we will distinguish between the two most common consensus algorithms: proof-of-work (PoW) and proof-of-stake (PoS). Other models, such as the proof-of-authority (PoA), delegated proof-of-stake (DPoS), directed-acyclical-graph (DAG) and others are either derivatives of the previous models or have not been explored in enough depth yet to properly identify possible vulnerabilities.\\

\subsection{Categorization}

Beyond the flaws in the theoretical model, we can consider the subject from a more comprehensive perspective. Indeed, the blockchain consensus is composed of three aspects: the algorithm implementing the theoretical model for the consensus, the parties which participate in the forming of the consensus and the governance mechanisms which dictate the rules of the consensus.\\

This means that a blockchain consensus platform could be attacked even if the theoretical model was perfectly secure. On the technical level, threats exist that are specific to blockchain platforms and exploits could be used to disrupt the access to the state in its full integrity.\\

Finally, the project governing a blockchain project can be disrupted in a number of ways, including a number of social engineering and manipulation vectors. This could lead to a situation where the rules of the consensus no longer follow the vision of the majority, leading from an exodus of the users from the platform and putting into question the consensus validity as a whole.\\

The paper will use these three different attack categories to differentiate the possible attacks on the consensus: theoretical attacks, technical attacks and social attacks.\\

\subsection{Security Criteria}

When analysing traditional information technology assets, we usually evaluate the severity of a risk in terms of the impact it has on confidentiality, integrity and availability. These three aspects can not be directly translated into the world of blockchain consensus. This paper proposes the following security criteria to make the evaluation of risks to blockchain consensus more meaningful: validity, efficiency and accessibility.\\

Validity refers to the desired immutability of the consensus state. Even though small changes to the recent history of a consensus state are normal, especially during periods of temporary contention, the risk that data from the established history can be changed after it's considered reliable should be negligible. As an example, if an attacker manages to change the recipient of the funds in a given transaction at a later point in time, the validity of the consensus is no longer given.\\

Efficiency refers to the ongoing process of finding consensus in the blockchain network and whether it can still be considered sufficient according to the established consensus rules. If the process is disrupted or producing incorrect or undesirable results, the efficiency at which the consensus algorithm can fulfill its original purpose diminishes. As such, once the efficiency of consensus finding is affected, the state changes it produces are less meaningful and the consensus can be considered broken on some level.\\

Accessibility refers to the ability of network participants to participate both in the consensus finding of the blockchain network, as well as the ability to use the blockchain network for its designed purpose. If a user is denied participation in the consensus finding or cannot transact on the network, the accessability of the consensus is lowered and no longer serves its original purpose.\\

\subsection{Analysis}

This paper uses a systematic definition of the information system security risk management domain, where a risk is considered as an event which leads to an impact on the security criteria of an asset. We alternatively use the term of exploit instead of event throughout this paper.\\

An exploit (or event) is defined as a threat agent using an attack method to exploit a vulnerability in an asset. Additionally, we propose a differentiation between the attack method and the attack that can be executed after a successful exploit, as these are often distinct in the area of blockchain.\\

Once the attacks have been elaborated, their impact on the security criteria of the consensus state are evaluated. Does the attack affect the validity, efficiency or the accessibility of the consensus state? Finally, we discuss a number of controls allowing for the mitigation of some of the impact on the security of the consensus state.\\

\section{Theoretical}

\subsection{Hash Power}

\subsubsection{Vulnerability}

One big vulnerability of any blockchain consensus protocol is the creation of an alternative history by a malicious actor which outgrows the currently shared network history, therefore allowing him to replace the consensus state with his own version of the history. For proof-of-work consensus protocols, this consists in the acquisition of more than fifty percent of hash power by one network participant. Even the acquisition of a significant minority hash power will enable certain attacks to be executed with a non-zero success rate.\\

\subsubsection{Threat}

The described vulnerability can be exploited by a malicious miner or by a mining pool operator. He can execute a number of attacks, but most of them fall into one of the following:
\begin{enumerate}
  \item Reverse own transactions
  \item Prevent transactions from gaining confirmations
  \item Prevent other miners from generating blocks
  \item Revert historical blocks
\end{enumerate}

\subsubsection{Attacks}

\subsubsection{Impact}

Once an attacker is able to execute double spend attacks, he undermines the basic premise and the main purpose of the blockchain network to find consensus on a common history of state changes. This affects the stability of the consensus.\\

At the same time, it allows the attacker to censor certain transactions, thus disrupting or even disabling the functionality of the blockchain network for certain participants. This affects the accessibility of the consensus.\\

The validity of the consensus is not affected; the attacker might have a certain leeway to produce a history desirable for him, but the history still has to be valid according to the consensus rules, otherwise his alternative history will be rejected by the network.\\

\subsubsection{Controls}

In order to avoid the acquisition of a significant portion of hash power by a single miner or, more accurately, by a single mining pool operator, we need to ensure a good distribution of hash power amongst mining pool operators. This can be done by implementing a simple software which automatically switches miners between different operators as soon as their accumulated hash power causes a relevant risk to the network.\\

In regards to double spend attacks, we can specifically reduce the risk for the recipients of any transaction by enforcing a wait for a certain amount of history to be created on the blockchain network after the transaction in question. The risk of a successful attack, especially with minority hash power, drops off exponentially with the amount of history that has to be overcome.\\

A more direct approach targeting double spend attacks consists of encouraging good broadcasting of all transactions by nodes and using network monitoring to detect double spends. In case of detection, the relevant transactions should be considered invalid until the consensus algorithm has resolved the conflict with a sufficiently high probability.\\

Solutions are:
\begin{enumerate}
  \item Automatic hash power rebalancing
  \item Monitoring double spends on the network
  \item Wait sufficient number of confirmations
  \item Introduce checkpoints
\end{enumerate}

\subsection{Block Withholding}

\subsubsection{Vulnerability}

The vulnerability targeted in the context of selfish mining is subtle. Nodes can choose what to propagate on the network; by withholding knowledge on a valid block, an informational asymmetry can be created on the network. This informational asymmetry presents the vulnerability.\\

\subsubsection{Threat}

A miner or mining pool operator can hide a valid block mined by himself from the network and start mining the next block in secret. This will give him a higher probability of extending the longest valid chain as compared to the rest of the network, thus earning him a disproportionate award.\\

\subsubsection{Attacks}

\subsubsection{Impact}

Just as the attack, the impact of the attack is subtle. The law of diminishing returns will lead to a market where inefficient miners will be forced to quit due to marginal profits. The selfish miner will be able to maintain his profitability longer, thus outcompeting other miners and gaining a disproportionate share of the network hash power.\\

Long term, the attack creates an unexpected barrier to entry into the mining market and thus affects the global accessibility of the consensus. At the same time, the attack jeopardizes both the stability and the validity of the consensus in a long term perspective.\\

\subsubsection{Controls}

One proposed solution to this weakness is the propagation of all competing branches of the blockchain and the random choice of which one to build on top of. This will make the threshold 25\% for everyone, thus eliminating a selfish miner's ability to take advantage of better propagation mechanisms. Another less symmetrical countermeasure is to detect selfish-mining by monitoring propagation behaviour and to refuse building on top of blocks discovered by selfish miners.\\

\subsection{Nothing-At-Stake}

\subsubsection{Vulnerability}

In the context of a proof-of-stake consensus algorithm, validators invest a negligible amount of resources into the generation of new blocks. This means that they can generate blocks for competing versions of the consensus state, thus weakening the consensus finding algorithm. This constitutes a serious vulnerability and is known as the nothing-at-stake problem.\\

\subsubsection{Threat}

Each validator on a proof-of-stake network is incentivized to create blocks for all competing histories of the consensus stake, as this allows them to guarantee the reward linked to the generation of blocks. If they wouldn't generate a block for a fork, but this fork would end up being the accepted history, they would lose their block awards on all other forks.\\

This allows attacks similar to the hash power attacks in a proof-of-work consensus, but at a much lower cost and without any risk. Practically no resources will be wasted in creating the competing history and all validators would extend the desired competing history. This means that even a low stake, and thus share in the block generation, would allow the attacker's fork to win eventually, as he would only generate blocks for the history he is interested in.\\

\subsubsection{Attacks}

\subsubsection{Impact}

The nothing-at-stake problem affects the consensus state drastically, as it undermines the whole point of the consensus algorithm: finding consensus on one state. Rather, in this case, all the states are extended and consensus is not found as they remain in conflict until a malicious user favours a specific state.\\

This can undermine the entire history of the consensus state and completely disrupt its stability. Each consensus finding process is following the consensus rules, but the attacker is the one who can choose the emerging consensus, thus also invalidating the validity of the consensus going forward.\\

\subsubsection{Controls}

Some networks instate a blacklist for validators who are caught creating blocks on an invalid or malicious blockchain history. This approach is imperfect and hard to enforce consistently, but it increases the costs of any attacks from negligible to a real amount. However, it doesn't fix the fundamental issue.\\

A more coherent solution is the introduction of so-called slashing conditions. In this model, each validator has to deposit a certain amount of funds in order to be elligible to generate blocks. If a validator continues extending the losing version of a blockchain, he will also lose part of his locked up funds. This will make each validator fully responsible for choosing the most correct fork of the history or incur a significant loss.\\

\subsection{Long Range}

\subsubsection{Vulnerability}

As costs for generating blocks in a proof-of-stake algorithm are negligible, any validator can generate as many blocks as he wants. This allows the recreation of part of all of the blockchain history. A malicious attacker can thus take over the network from any point of the blockchain history where he held a sufficient amount of block generation power to overpower the competing validators.\\

\subsubsection{Threat}

A malicious validator only needs to acquire a sufficient stake on the network once in order to be able recreate the entire history from that point forward. This includes the ability to include conspiring validators a posteriori.\\

The attacks they can then execute are many, as they can write the entire history according to their whim, excluding and thus censoring any transaction they want and double spending any transaction that was outgoing from their funds.\\

\subsubsection{Attacks}

\subsubsection{Impact}

A long range attack can invalidate any history of the consensus state and create new state for all addresses controlled by the attacker. It thus completely crushes the stability of the consensus.\\

Additionally, it allows the attackers to censor and thus disrupt accessibility of the consensus, making the blockchain network unusable for its designed purpose.\\

\subsubsection{Controls}

One approach is to introduce checkpoints or to limit the number of blocks that can be rewritten, thus creating an upper bound for how much of the history can be rewritten. This creates the concept of finality, which means that a block can never be changed after a certain point in time and can thus be considered final.\\

Proper bootstrapping of nodes and catching up when falling too far behind the network consensus state still leaves a vulnerability in that case, though. There needs to be a trusted mechanism to transmit the correct history out-of-band of the blockchain. This concept is called weak subjectivity.\\

\section{Technical}

\subsection{Network Partitioning}

\subsubsection{Vulnerability}

Blockchain networks generally rely on rudimentary peer-to-peer gossip protocols to propagate messages, such as blocks and transactions, through the network. By isolating nodes or by creating artificial subgroups of nodes, the view of the consensus state can be manipulated.\\

\subsubsection{Threat}

A number of attacks are possible by exploiting the network topology of blockchain networks.\\

In an eclipse attack, the attacker monopolizes all connections of a victim node, therefore controlling the view the node has on the consensus state. This can be exploited for a number of attacks:\\
\begin{enumerate}
  \item Engineer block races, costing the attacked miner resources.
  \item Splitting mining power, enabling easier 51\% attacks.
  \item Supporting selfish mining, costing the attacked and gaining more.
  \item 0-confirmation double spend.
  \item N-confirmation double spend.
\end{enumerate}

\subsubsection{Attacks}

\subsubsection{Impact}

As network partitioning attacks are based on

\subsubsection{Controls}

\subsection{Delay}

\section{Social}

\subsection{Arbitrary Content}

\subsubsection{Vulnerability}

Many blockchain platforms allow nodes to include arbitrary data into the transaction, which constitutes a clear vulnerability.\\

\subsubsection{Threat}

A malicious actor, of any form, who would desire to disrupt a blockchain network could inject illegal material into a blockchain, which would then exist in immutable form for the lifetime of the consensus state. This same actor could then use the existence of illegal material in the consensus state to scare other people away from the platform.\\

\subsubsection{Impact}

Having illegal material embedded into the blockchain is not any more of a legal issue that having hidden material in an image you load from a website, which is why the attack is part of the social category. However, scaring people away from a blockchain network affects the accessibility of the consensus.\\

\subsection{Community Manipulation}

\subsubsection{Vulnerability}

Blockchain networks are based on consensus between network participants that is reflected in the consensus rules and reached through the consensus algorithm. However, if you manage to influence the group of people who contribute to the consensus finding (miners or validators), you can influence the consensus rules and changes to them.\\

\subsubsection{Threat}

While blockchain consensus protocols allow decentralized consensus according to predefined rules, these consensus rules can always be changed by changing the code of the nodes participating in the network. A malicious party can use censorship of media channels and other social media manipulation tactics to influence the perceived consensus of the community and thus counteract the initially intended consensus rules.\\

\subsubsection{Impact}

Changes to consensus rules are difficult to achieve, but as past events have shown, they are not impossible. Depending on the original design, a change in the consensus rules has a high probability of affecting the validity of the consensus after the change has been done.\\

Additionally, consensus rules can be drastic enough to affect the past blockchain history, even though it's unlikely that the impact will be big here. Finally, by implementing rules that many network participants don't agree with or which affect the functioning of the blockchain platform, the accessibility of the consensus can be hampered as well.\\

\subsubsection{Controls}

As a first step, a consensus platform needs clear values to govern the surrounding software project and a transparent decision-making process that can be audited by anyone. It is thus important to ensure that all big discussion platforms are democratically governed.\\

Alternatively, an official discussion platform that is governed through decisions made at the heart of the consensus itself would make every network participant a true stakeholder in how the community and the project is run, thus avoiding any form of hostile takeover to influence processes.\\

\subsection{Bribes \& Refunds}

\subsubsection{Vulnerability}

Consensus is achieved on blockchain networks through the use of economic incentives. If a malicious actor tries to act against the consensus rules, he will usually incur a more or less severe economic punishment. However, the fact that such a loss can be countered by out-of-band payments makes the system vulnerable to bribery.\\

\subsubsection{Threat}

\subsubsection{Impact}

\subsubsection{Controls}

\section{Case Study}

\subsection{Analysis}

\subsection{Evaluation}

\subsection{Mitigation}

\section{Conclusion}

\newpage
\begin{thebibliography}{9}

\bibitem{bitcoin}
Satoshi Nakamoto,
"Bitcoin: A Peer-to-Peer Electronic Cash System",
https://bitcoin.org/bitcoin.pdf,
October 2018.

\end{thebibliography}

\newpage
\appendix

\section{Risks}

\begin{tabular}{| l | l | l |}
  \hline
  \multicolumn{3}{| l |}{A.1 Theoretical Risks}\\
  \hline
  \multicolumn{3}{| l |}{A.2 Technical Risks}\\
  \hline
  \multicolumn{3}{| l |}{A.3 Social Risks}\\
  \hline
\end{tabular}


\end{document}
