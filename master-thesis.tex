\documentclass[11pt,a4paper,draft]{article}

\title{Risk Management for Blockchain Consensus}
\author{Max Wolter}
\date{June 2018}

\begin{document}

\maketitle

\begin{abstract}

\end{abstract}

\section{Introduction}

The concept of Bitcoin as a "peer-to-peer electronic cash system" was introduced by an anonymous author under the pseudonym of Satoshi Nakamoto in October 2008. Since then, the potential of the underlying blockchain technology to serve as a trustless mechanism to find consensus on the state of an underlying system has spread to many applications.\\

The main value proposition of a blockchain platform is its ability to ensure the integrity of the consensus in a verifiable and transparent manner. While the software and business ecosystem developed around a blockchain project are susceptible to the same risks as typical software and business projects, the core platform has its own, unique set of risks.\\

This paper explores the various aspects around blockchain consensus systems that are required to guarantee the continued security of the system state.\\

\section{Consensus}

\subsection{Weaknesses}

When identifying risks, the usual approach dictates that we identify assets which are vulnerable to use them as a basis for the analysis. For a blockchain network, this proposition is a difficult one; the network is composed by thousands of nodes which communicate in a peer-to-peer fashion. While each node can be compromised in its own right, there is no single tangible asset or actor that impacts the security of the network state as a whole.\\

However, we could define the consensus itself as an abstract concept which needs to be protected. This leads us to the first family of weaknesses, where we look for flaws in the theoretical model used for the blockchain consensus. It requires exploration of the possible exploits in the fields of mathematics, more specifically game theory, and information science, in particular distributed systems.\\

Beyond these obvious flaws, we can consider the subject from a more comprehensive perspective. Indeed, the blockchain consensus is composed of three aspects: the algorithm implementing the theoretical model for the consensus, the parties which participate in the forming of the consensus and the governance mechanisms which dictate the rules of the consensus.\\

This means that a blockchain consensus platform could be attacked even if the theoretical model was perfectly secure. On the technical level, threats exist that are specific to blockchain platforms and exploits could be used to disrupt the access to the state in its full integrity.\\

Finally, the project governing a blockchain project can be disrupted in a number of ways, including a number of social engineering and manipulation vectors. This could lead to a situation where the rules of the consensus no longer follow the vision of the majority, leading from an exodus of the users from the platform and putting into question the consensus validity as a whole.\\

The paper will use these three different attack categories to differentiate the possible attacks on the consensus: theoretical attacks, technical attacks and social attacks.\\

\subsection{Impact}

When analysing traditional information technology assets, we usually evaluate the impact of an exploited weakness in terms of the impact it has on confidentiality, integrity and availability of information. In a blockchain network, these three aspects are slightly different.\\

Transparency is a cornerstone of blockchain technology and usually, all participants of the network have access to all the data in order to form consensus of the state. While exceptions exist, confidentiality is not an adequate measurement of the impact of disrupted consensus.\\

When talking about integrity, blockchain technology theoretically guarantees that the consensus state is reliable. However, if there was to be a weakness to the consensus algorithm, the integrity of the consensus itself could be questioned. As such, we can interpret integrity to mean the validity of the consenus, and we will use that nomencloture to clearly differentiate.\\

Finally, the blockchain network offers a highly redundant representation of the consensus state, which makes it readily available under most circumstances. However, it is possible to disrupt the access of certain nodes or even groups of nodes to the correct consensus state or to the functionaliny of the platform. We will term this impact as accessibility to make it clear it's different from the usual availability criterium.\\

In addition to the two aspects of impact that find an interpretation in the context of blockchain technology, which are how valid is the consensus state and how sure can I be that I am seeing the correct consensus state, we can add another factor that represents a time dimension to the equation. How reliable will the consensus state be in the future? We term this as stability.\\

We therefore propose to estimate the impact of a consensus weakness in terms of its effects on the validity, accessability and stability of the consensus state.\\

\section{Reach}

Next to the classification of the impact, we can also establish a reach for various exploits of the consensus state weaknesses. This can happen in two dimensions: spread and degree.\\

The spread of an exploit's impact talks about the subset of nodes affected. It can either be a single victim (one node or user), a group of victims (a certain demographic or a segment of the network) or all of the nodes.\\

The degree of an exploit's impact talks about the level that each affected node is impacted. It can be partial (only some operations don't work, or only some nodes of the group are affected) or complete (all operations stop working, or all nodes of the group are equally affected).\\

\section{Game Theory}

51\% Attack

Selfish Mining

Long Range Attack

(Censorship, Economic Equilibrium, Double Spend)


\section{Participation}

Eclipse Attack

Partition Attack

Delay Attack

Illegal Material

\section{Governance}

\subsection{Community Sentiment}

\subsection{Software Code}

\subsection{Project Roadmap}

\section{Interdependencies}

\section{Case Study}

\subsection{Analysis}

\subsection{Evaluation}

\subsection{Mitigation}

\section{Conclusion}

\cite{item}

\begin{thebibliography}{9}

\bibitem{item}
author,
"name",
link,
date.

\end{thebibliography}

\end{document}
