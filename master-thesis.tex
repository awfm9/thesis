\documentclass[11pt,a4paper,draft]{article}

\usepackage[utf8]{inputenc}
\usepackage[english]{babel}
\usepackage[nottoc,numbib]{tocbibind}

\setcounter{secnumdepth}{3}
\setcounter{tocdepth}{3}

\title{Risk Management for Blockchain Consensus}
\author{Max Wolter}
\date{June 2018}

\begin{document}

\maketitle
\begin{abstract}
\end{abstract}
\newpage

\tableofcontents
\newpage

\section{Introduction}

The concept of a blockchain consensus network was introduced in the form of Bitcoin in 2008 \cite{bitcoin}. Since then, the potential of the underlying blockchain technology to serve as a trustless mechanism to find consensus on the state of an underlying system has spread to many applications beyond the maintenance of a ledger of account balances.\\

The main value proposition of a blockchain platform is its ability to ensure the security of the consensus state in a verifiable and transparent manner. While the software and business ecosystem developed around a blockchain project are susceptible to the same information security risks as typical software and business projects, the core platform has its own unique security properties.\\

This paper explores the applicability of the traditional information security risk management approach to the context of blockchain consensus. It makes an attempt to integrate the current research litterature in the blockchain space with the more mature sphere of information security risk management.\\

\section{Framework}

\subsection{Asset Identification}

When identifying risks, the usual approach dictates that we identify assets which are vulnerable and use them as a basis for the analysis. For a blockchain network, this proposition is a difficult one; the network is composed by thousands of nodes which communicate in a peer-to-peer fashion. While each node can be compromised in its own right, there is no single tangible asset or actor that impacts the security of the consensus state as a whole. The same can be said for the periphery, consisting of infrastructure, tools and solutions built around the blockchain network.\\

Our approach posits that we can define the abstract concept of the consensus state itself as the asset which needs to be protected. In that context, we can focus on vulnerabilities that are unique to blockchain networks. We can establish our own security criteria which should be fulfilled in order for the consensus state to be considered secure. At the end, can evaluate the potential risks for the consensus state in terms of impact on these criteria.\\

There are various algorithms used to achieved consensus in blockchain networks. Where the type of consensus algorithm is relevant, we will focus on the two most common algorithms: proof-of-work (PoW) and proof-of-stake (PoS). Other models, such as the directed-acyclical-graph (DAG) approach or derivatives of the PoS approach, have not been explored in sufficient depth in scientific literature to provide a meaningful analysis.\\

\subsection{Security Criteria}

As alluded to earlier, the usual security criteria of confidentiality, integrity and availability are not a good fit for the abstract concept of the consensus state in a blockchain network. We can, however, establish our own security criteria which make sense in the context of blockchain networks. The foundation for these criteria should be found in the basic process of consensus finding on the network.\\

A first step in establishing a shared consensus state between the participants of a blockchain network is their ability to participate in the consensus algorithm. When network participants cannot play their desired role in the process, the transparency \& openess of the blockchain network is diminished. We call this first security criterium the accessibility of the consensus state.\\

A second step in the process of finding consensus between network participants is the correct execution of the consensus algorithm over time. When weaknesses undermine the immutability or reliability of the blockchain history, a core property of the blockchain network is broken. We call this second security criterium the validity of the consensus state.\\

Finally, the blockchain network should make optimal use of the resources at its disposal and treat all participants equally. If there are disruptions to the economic equilibrium used to enforce the consensus rules, the security guarantees of the blockchain network are diminished. We call this third security criterium the efficiency of the consensus state.\\

\subsection{Vulnerability Categories}

The most obvious category of vulnerabilities on blockchain platforms are based on the flaws that are inherent in the theoretical models of consensus finding. In this category, we explore possible exploits that rely on work in the field of mathematics, more specifically game theory, and information science, in particular in regards to distributed systems. All of these risks are confined to the model of consensus. We will define these intrinsic aspects as theoretical vulnerabilities.\\

Beyond these, however, there are technical means to disrupt the consensus state that are outside of the direct theoretical model. Certain attacks can exploits aspects of the peer-to-peer topology underlying the blockchain network in order to disrupt or manipulate the consensus state. We will describe these out-of-bound aspects as technical vulnerabilities.\\

Last but not least, a blockchain network is still governed by software that encodes the consensus rules, and this software is still written by humans. This means that the direction of a blockchain project, including the evolution of the consensus rules, is subject to the weaknesses of human governance and community factors. We call these aspects the social vulnerabilities.\\

\subsection{Risk Analysis}

Our method for analysing risks is based on a systemic definition of the information security risk management domain. This nomencloture defines a risk as an event, occurring with a certain probability, leading to a certain impact on the security criteria of our asset. An event, in return, is a threat agent using an attack method to exploit a vulnerability.\\

In this paper, we propose to extend the framework slightly to make a clear distinction between the attack method and the attacks themselves. Indeed, one attack method applied to one vulnerability can open the door to the execution of a variety of attacks in the context of blockchain. In general, this means that we address the vulnerability directly, rather than handling each attack on its own.\\

We therefore establish the following definition as the basis for our risk analysis: a risk is the event with certain probability of a threat agent using an attack method to exploit a vulnerability of the blockchain network in order to execute an attack, impacting the validity, accessibility or efficiency of the consensus state.\\

\subsection{Mitigation Controls}

In the context of blockchain networks, the mere exploitation of a vulnerability can lead to the collapse of the security criteria of the consensus state, thus making the blockchain network useless for its intended purposes, even if an attack isn't necessarily executed. In general, this means that controls need to mitigate vulnerabilities or attack methods, rather than just avoid attacks.\\

\section{Theoretical Risks}

\subsection{Hash Power}

The security of proof-of-work blockchains relies on the the miners, which compete in a lottery to generate the next block in order to obtain the related reward. Through this design, all miners are incentivized to work on the latest valid version of the blockchain history, as they would otherwise run a higher risk of not obtaining said reward.\\

However, this also presents a significant vulnerability for the blockchain network as a whole. If a single miner manages to overwhelm all other miners combined with his hash power, he can dictate the history of the blockchain and generate all blocks by himself. Obtaining a majority of the hash power thus opens the door to a number of possible attacks which break the security guarantees of the consensus state.\\

Due to the nature of the proof-of-work algorithm, which is a memoryless process, there is also significant variance. This means that a number of attacks are even possible with a relevant minority portion of hash power on the network, allowing a malicious miner to execute them with a non-zero probability of success.\\

In general, the threat agent in this case is a miner or the operator of a mining pool, as control over the creation of the blocks is necessary in order to use the vulnerability.\\

Once an attacker is able to interfer with the basic mechanism of consensus finding on the blockchain network, all security criteria are affected to a certain degree. The history can be changed, thus affecting validity of the consensus state. The blocks of other participants can be invalidated, thus decreasing the efficiency of consensus finding. Finally, transactions can be censored, thus disrupting accessibility to the consensus functionality.\\

\subsubsection{Double Spending}

The double spend attack breaks the most fundamental purpose of blockchain consensus, but only does so for specific targeted transactions. It allows the attacker to convey one version of consensus state to the victim, where the victim receives the desired funds of a transaction. This consensus state is later replace it with another version of the consensus state, which keeps the funds in the attacker's pockets.\\

A simple version of this attack is the zero-confirmation double spend, where the malicious miner simple includes a different version of the transaction in his block than the one submitted to the network when paying the victim. This can be done by either spreading both versions to the network or by withholding the second version until the payment has been accepted.\\

The attack can be counter-acted in two ways. First, the nodes on the blockchain network can be configured to relay conflicting transactions in the memory pool, thus allowing the detection of double spending. The same result can be achieved by creating a network of observers on the blockchain network to achieve a good view of all transactions propagated in different parts of the network.\\

A simpler solution, which however affects the speed of being able to accept payments, involves waiting for the pending transaction to be confirmed, or even to be confirmed several times by the desired number of sequential blocks. As the probability of being able to outperform the whole network diminishes exponentially with the number of confirmations, the adequate number can be derived from a simple formula.\\

However, in the case that an attacker obtains an absolute majority of hashing power of the network, none of these strategies are effective as he can rebuild the desired length of blockchain history for the consensus state with a one hundred percent probability. In that case, the only available control would be a mechanism to balance hash power across mining pools through automated software.\\

\subsubsection{Censoring Transactions}

A miner can decide which transactions to include or exclude in each block. This naturally gives him the ability to throttle transactions of certain market participants. Once his hashing power reaches absolute majority, he would even be able to build the blockchain history on his own, allowing him to censor all transactions and all blocks by network participants other than himself.\\

Once more, the primary goal should be to avoid the acquisition of a hash power majority by any party, possibly through auto-balancing of pools. Another approach to avoid censorship would be the introduction of anonymous transactions, which makes it at least impossible to target specific network participants.\\

\subsubsection{Rewriting History}

If a miner were to acquire a majority of hash power on the network, he could execute an attack to damage the blockchain network viability itself by simply rewriting the history and making it invalid going far back into history. This would destroy the value proposition of the platform and spell its doom.\\

In order to avoid this possibility, especially for blockchains that are vulnerable due to a low amount of total hash power, checkpoints can be introduced into the software. A rewrite of history that does not include certain checkpoint blocks will simply be considered invalid by nodes on the network, thus hindering rewrites beyond a certain point.\\


\subsection{Block Withholding}

Every miner participating in a proof-of-work consensus finding algorithm relies on the propagation of valid blocks through the network in order to generate a state transition for the latest valid consensus state. The theoretical model assumes that all miners will freely share newly found blocks in order to obtain their reward.\\

However, the option to withhold a discovered block and thus hide it from the network is a theoretical possibility. Depending on the motivation of an attacker, this opens up a vulnerability on the consensus state. An attacker can withhold a block to gain profit in some other way, while still conserving a significant probability of obtaining the block reward.\\

The vulnerability has the potential to disrupt the economic equilibrium between miners, by either letting them gain an extra profit, or by causing an extra loss to other miners, thus reducing the efficiency of the consensus finding. Additionally, it has potential to replace part of the blockchain history by a previously computed alternative hidden version, thus affecting validity as well.\\

\subsubsection{Finney Attack}

The Finney attack is a more sophisticated version of a double spend that does not rely on having a significant amount of hash power on the network. Instead, the miner waits to generate a block for whatever interval it takes and only starts the attack subsequently. Once the block is found, it's withheld from the network and thus hidden. It includes a transaction that sends certain funds from the attacker to the victim.\\

The attacker now propagates a conflicting transaction that sends the funds to the victim as part of the transaction. As soon as the desired exchange has occurred in both directions, the attacker propagates his hidden block and reclaims the funds. It should be noted that the attacker risks another block being found before the attack was executed, thus risking to lose the block reward.\\

The attack rewrites a small part of the blockchain history and can thus be considered to affect the validity of the consensus state. It can, however, be easily mitigated by introducing a waiting period for a certain amount of blocks, which makes it probabilistically unlikely that the attacker can derive an economic advantage from the execution on the attack, due to the lost block reward.\\

\subsubsection{Block Discarding}

The block discarding attack can be used to economically damage a mining pool. A lot of mining pools reward miners that contribute their work to the pool in proportion to the work completed. In order to get rewards, they submit proofs of work at a significantly smaller level of difficulty than the one required for a valid block. These so-called shares are only useful to estimate the work accomplished by each miner.\\

As they are found significantly more often than a valid block, a malicious participant of the mining pool can submit them and receive most of his mining reward. When he finds a valid block, however, he can simply discard it instead of submitting it to the pool operator. This will lead to a significant loss to the pool operator, while being negligible for the miner himself.\\

Some pool operators mitigate the risk by using a reward structure that attributes extra rewards on submission of a valid block, thus making the loss non-significant for the malicious miner, while at the same time making it less costly for the mining pool operator.\\

\subsubsection{Selfish Mining}

The most sophisticated block withholding attack is the selfish mining strategy. A malicious miner who finds a block can simple withhold it from the network in order to gain a headstart looking for the next valid block. Once the network finds a block, he can simply release his block to contend for the block reward. However, he will significantly increase his own likelihood of finding the next valid block, which will allow him to propagate both blocks and thus obtain two block rewards at once: one at the normal probability and one at an increased probability.\\

One counter to this strategy is to simply accept the economic incentive of employing the strategy as part of the consensus framework. One downside would be that miners with better connectivity would stand a better chance of contending block rewards when letting a block out of hiding. This can be countered by accepting all blockchain heads an randomly choosing one to mine on top of, giving all network participants an equal chance to content.\\

\subsection{Cheap Validation}

In the context of a proof-of-stake consensus algorithm, validators invest a negligible amount of resources into the generation of new blocks. This means that they can generate blocks for competing versions of the consensus state, thus weakening the consensus finding algorithm. This constitutes a serious vulnerability and is known as the nothing-at-stake problem.\\

Each validator on a proof-of-stake network is incentivized to create blocks for all competing histories of the consensus stake, as this allows them to guarantee the reward linked to the generation of blocks. If they wouldn't generate a block for a fork, but this fork would end up being the accepted history, they would lose their block awards on all other forks. Validating on all forks is a zero-risk proposition.\\

At the same time, the low cost of generating blocks allows anyone to produce a long history of blocks in a short amount of time. Practically no resources will be wasted in creating the competing history and all validators would extend the desired competing history. This means that even a low stake, and thus share in the block generation, would allow the attacker's fork to win eventually, as he would only generate blocks for the history he is interested in.\\

On a basic level, this undermines the efficiency of finding consensus on the blockchain history completely, as any number of competing histories will exist and conflict between them is not resolved. Additionally, a complete rewrite of the history undermines the validity of the consensus state.\\

\subsubsection{Nothing-At-Stake}

While not strictly an attack with a specific victim, the nothing-at-stake problem still represents a significant risk for the security of the blockchain network, as it undermines the fundamental functioning of the consensus finding algorithm. Validators extend all branches of the blockchain history, thus never resolving conflicts and not fulfilling the basic premise of a blockchain network to have a single shared consensus state.\\

\subsubsection{Long-Range Attack}

%%%%%%%%%%%%%%%%%%%%%%%%

The nothing-at-stake problem affects the consensus state drastically, as it undermines the whole point of the consensus algorithm: finding consensus on one state. Rather, in this case, all the states are extended and consensus is not found as they remain in conflict until a malicious user favours a specific state.\\

This can undermine the entire history of the consensus state and completely disrupt its stability. Each consensus finding process is following the consensus rules, but the attacker is the one who can choose the emerging consensus, thus also invalidating the validity of the consensus going forward.\\


Some networks instate a blacklist for validators who are caught creating blocks on an invalid or malicious blockchain history. This approach is imperfect and hard to enforce consistently, but it increases the costs of any attacks from negligible to a real amount. However, it doesn't fix the fundamental issue.\\

A more coherent solution is the introduction of so-called slashing conditions. In this model, each validator has to deposit a certain amount of funds in order to be elligible to generate blocks. If a validator continues extending the losing version of a blockchain, he will also lose part of his locked up funds. This will make each validator fully responsible for choosing the most correct fork of the history or incur a significant loss.\\

As costs for generating blocks in a proof-of-stake algorithm are negligible, any validator can generate as many blocks as he wants. This allows the recreation of part of all of the blockchain history. A malicious attacker can thus take over the network from any point of the blockchain history where he held a sufficient amount of block generation power to overpower the competing validators.\\

A malicious validator only needs to acquire a sufficient stake on the network once in order to be able recreate the entire history from that point forward. This includes the ability to include conspiring validators a posteriori.\\

The attacks they can then execute are many, as they can write the entire history according to their whim, excluding and thus censoring any transaction they want and double spending any transaction that was outgoing from their funds.\\

A long range attack can invalidate any history of the consensus state and create new state for all addresses controlled by the attacker. It thus completely crushes the stability of the consensus.\\

Additionally, it allows the attackers to censor and thus disrupt accessibility of the consensus, making the blockchain network unusable for its designed purpose.\\

One approach is to introduce checkpoints or to limit the number of blocks that can be rewritten, thus creating an upper bound for how much of the history can be rewritten. This creates the concept of finality, which means that a block can never be changed after a certain point in time and can thus be considered final.\\

Proper bootstrapping of nodes and catching up when falling too far behind the network consensus state still leaves a vulnerability in that case, though. There needs to be a trusted mechanism to transmit the correct history out-of-band of the blockchain. This concept is called weak subjectivity.\\

\section{Technical Risks}

\subsection{Network Partitioning}

\subsubsection{Vulnerability}

Blockchain networks generally rely on rudimentary peer-to-peer gossip protocols to propagate messages, such as blocks and transactions, through the network. By isolating nodes or by creating artificial subgroups of nodes, the view of the consensus state can be manipulated.\\

\subsubsection{Threat}

A number of attacks are possible by exploiting the network topology of blockchain networks.\\

In an eclipse attack, the attacker monopolizes all connections of a victim node, therefore controlling the view the node has on the consensus state. This can be exploited for a number of attacks:\\
\begin{enumerate}
  \item Engineer block races, costing the attacked miner resources.
  \item Splitting mining power, enabling easier 51\% attacks.
  \item Supporting selfish mining, costing the attacked and gaining more.
  \item 0-confirmation double spend.
  \item N-confirmation double spend.
\end{enumerate}

\subsubsection{Attacks}

\subsubsection{Impact}

As network partitioning attacks are based on

\subsubsection{Controls}

\subsection{Message Delaying}

\section{Social}

\subsection{Content Insertion}

\subsubsection{Vulnerability}

Many blockchain platforms allow nodes to include arbitrary data into the transaction, which constitutes a clear vulnerability.\\

\subsubsection{Threat}

A malicious actor, of any form, who would desire to disrupt a blockchain network could inject illegal material into a blockchain, which would then exist in immutable form for the lifetime of the consensus state. This same actor could then use the existence of illegal material in the consensus state to scare other people away from the platform.\\

\subsubsection{Impact}

Having illegal material embedded into the blockchain is not any more of a legal issue that having hidden material in an image you load from a website, which is why the attack is part of the social category. However, scaring people away from a blockchain network affects the accessibility of the consensus.\\

\subsection{Community Manipulation}

\subsubsection{Vulnerability}

Blockchain networks are based on consensus between network participants that is reflected in the consensus rules and reached through the consensus algorithm. However, if you manage to influence the group of people who contribute to the consensus finding (miners or validators), you can influence the consensus rules and changes to them.\\

\subsubsection{Threat}

While blockchain consensus protocols allow decentralized consensus according to predefined rules, these consensus rules can always be changed by changing the code of the nodes participating in the network. A malicious party can use censorship of media channels and other social media manipulation tactics to influence the perceived consensus of the community and thus counteract the initially intended consensus rules.\\

\subsubsection{Impact}

Changes to consensus rules are difficult to achieve, but as past events have shown, they are not impossible. Depending on the original design, a change in the consensus rules has a high probability of affecting the validity of the consensus after the change has been done.\\

Additionally, consensus rules can be drastic enough to affect the past blockchain history, even though it's unlikely that the impact will be big here. Finally, by implementing rules that many network participants don't agree with or which affect the functioning of the blockchain platform, the accessibility of the consensus can be hampered as well.\\

\subsubsection{Controls}

As a first step, a consensus platform needs clear values to govern the surrounding software project and a transparent decision-making process that can be audited by anyone. It is thus important to ensure that all big discussion platforms are democratically governed.\\

Alternatively, an official discussion platform that is governed through decisions made at the heart of the consensus itself would make every network participant a true stakeholder in how the community and the project is run, thus avoiding any form of hostile takeover to influence processes.\\

\subsection{Bribes \& Refunds}

\subsubsection{Vulnerability}

Consensus is achieved on blockchain networks through the use of economic incentives. If a malicious actor tries to act against the consensus rules, he will usually incur a more or less severe economic punishment. However, the fact that such a loss can be countered by out-of-band payments makes the system vulnerable to bribery.\\

\subsubsection{Threat}

\subsubsection{Impact}

\subsubsection{Controls}

\section{Case Study}

\subsection{Analysis}

\subsection{Evaluation}

\subsection{Mitigation}

\section{Conclusion}

\newpage
\begin{thebibliography}{9}

\bibitem{bitcoin}
Satoshi Nakamoto,
"Bitcoin: A Peer-to-Peer Electronic Cash System",
https://bitcoin.org/bitcoin.pdf,
October 2018.

\end{thebibliography}

\newpage
\appendix

\section{Risks}

\begin{tabular}{| p{3cm} | p{5cm} | p{5cm} |}
  \hline
  \multicolumn{3}{| l |}{\textbf{A.1 Theoretical Risks}}\\
  \hline
  \multicolumn{3}{| l |}{A.1.1 Hash Power Attacks}\\
  \hline
    \textit{Majority Attack} & Reverse own transactions, prevent transactions, prevent block generation, revert blocks & Validity, Efficiency, Accessibility\\
  \hline
    \textit{Minority Attack} & Reverse own transactions, throttle transactions & Validity, Accessibility\\
  \hline
  \multicolumn{3}{| l |}{A.1.2 Block Withholding Attacks}\\
  \hline
    \textit{Finney Attack} & Reverse own transactions & Validity\\
  \hline
    \textit{Block Discarding} & Waste network resources & Efficiency\\
  \hline
    \textit{Selfish Mining} & Waste network resources & Efficienty\\
  \hline
  \multicolumn{3}{| l |}{A.1.3 Cheap Validation Attacks}\\
  \hline
    \textit{Fork Staking} & Undermine conflict resolution & Efficiency\\
  \hline
    \textit{Long Range Attack} & Reverse own transactions, invalidate past transactions & Validity, Accessibility\\
  \hline
  \multicolumn{3}{| l |}{A.1.4 Long Range Attacks}\\
  \hline
  \multicolumn{3}{| l |}{\textbf{A.2 Technical Risks}}\\
  \hline
  \multicolumn{3}{| l |}{A.2.1 Network Partitioning Attacks}\\
  \hline
  \multicolumn{3}{| l |}{A.2.2 Node Isolation Attacks}\\
  \hline
  \multicolumn{3}{| l |}{A.2.3 Communication Delay Attacks}\\
  \hline
  \multicolumn{3}{| l |}{\textbf{A.3 Social Risks}}\\
  \hline
  \multicolumn{3}{| l |}{A.3.1 Arbitrary Content Attacks}\\
  \hline
  \multicolumn{3}{| l |}{A.3.2 Community Manipulation Attacks}\\
  \hline
  \multicolumn{3}{| l |}{A.3.3 Governance Structure Attacks}\\
  \hline
\end{tabular}


\end{document}
