\documentclass[11pt,a4paper,draft]{article}

\title{Risk Management for Blockchain Consensus}
\author{Max Wolter}
\date{June 2018}

\begin{document}

\maketitle

\begin{abstract}

\end{abstract}

\section{Introduction}

The concept of Bitcoin as a "peer-to-peer electronic cash system" was introduced
by an anonymous author under the pseudonym of Satoshi Nakamoto in October 2008.
Since then, the potential of the underlying blockchain technology to serve as
a trustless mechanism to find consensus on the state of an underlying system has
found many applications.\\

The main value proposition of a blockchain platform is its ability to ensure
the integrity of the consensus in a verifiable and transparent manner. While the
software and business ecosystem developed around a blockchain project are
susceptible to the same risks as typical software and business projects, the
core platform has its own, unique set of risks.\\

This paper explores the various aspects of consensus that are required to
guarantee the continued security of the system state of a blockchain network.\\

\section{Consensus}

When identifying risks, the usual approach dictates that assets which are
vulnerable are identified and analyzed. For a blockchain network, this
proposition is a more difficult one; the network is constituted by thousands of
nodes which communicate in a peer-to-peer fashion. There is no single tangible
asset or actor that impacts the security of network state.\\

Rather, the consensus as an abstract concept is the asset that needs to be
protected. At first glance, we are therefore trying to protect a theoretical
model from possible flaws. This necessitates the mathematical exploration of
weaknesses in the model and ways to safeguard against them.\\

However, considering the subject from a more comprehensive perspective, a
blockchain network is governed by the rules encoded in the software of the
project. It is thus possible to affect the reliability of the consensus by
modifying the software and changing the rules. This can be done by interfering
with the project governance directly, or by manipulating it through social
means.\\

We therefore identify three categories of attacks on the consensus of a
blockchain system: theoretical attacks targeting the game theory model of the
consensus algorithm, organizational attacks targeting the governance of the
project and social attacks targeting the community around the project.\\

\subsection{Game Theory}

\subsection{Governance}

\subsection{Community}

\section{Theoretical}

\subsection{Threats}

\subsection{Impacts}

\subsection{Controls}

\section{Organizational}

\subsection{Threats}

\subsection{Impacts}

\subsection{Controls}

\section{Social}

\subsection{Threats}

\subsection{Impacts}

\subsection{Controls}

\section{Interdependencies}

\section{Case Study}

\subsection{Analysis}

\subsection{Evaluation}

\subsection{Mitigation}

\section{Conclusion}

\cite{item}

\begin{thebibliography}{9}

\bibitem{item}
author,
"name",
link,
date.

\end{thebibliography}

\end{document}
